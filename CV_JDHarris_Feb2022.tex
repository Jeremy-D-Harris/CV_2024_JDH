        % vim: set textwidth=120:
        
        % Example CV based on the 1.5-column-cv template. Main features:
        % * uses the Tex Gyre Heros font and icons from the Europass template;
        % * uses colours of the Europass colour palette for styling;
        % * displays name, date, and current page number in header and footer;
        % * doesn't include a photo.
        \documentclass[a4paper,10pt]{article}
        
        
        % package imports
        % ---------------
        
        \usepackage[british]{babel} % for correct language and hyphenation and stuff
        \usepackage{calc}           % for easier length calculations (infix notation)
        \usepackage{enumitem}       % for configuring list environments
        \usepackage{fancyhdr}       % for setting header and footer
        \usepackage{fontspec}       % for fonts
        \usepackage{geometry}       % for setting margins (\newgeometry)
        \usepackage{graphicx}       % for pictures
        \usepackage{lastpage}       % for getting total number of pages (LastPage)
        \usepackage{microtype}      % for microtypography stuff
        \usepackage{xcolor}         % for colours
        \usepackage{hyperref}     % jdh added to template -- this package allows you to set links with a description as well as add bare urls to your document.
        \hypersetup{
            colorlinks=true,
            linkcolor=blue,
            filecolor=magenta,      
            urlcolor=cyan,
        }
        
        
        % margin and column widths
        % ------------------------
        
        % margins
        \newgeometry{includehead,includefoot,left=12mm,right=15mm,top=12mm,bottom=12mm}
        
        % width of the gap between left and right column
        \newlength{\cvcolumngapwidth}
        \setlength{\cvcolumngapwidth}{3mm}
        
        % left column width
        \newlength{\cvleftcolumnwidth}
        \setlength{\cvleftcolumnwidth}{47mm}
        
        % right column width
        \newlength{\cvrightcolumnwidth}
        \setlength{\cvrightcolumnwidth}{\textwidth-\cvleftcolumnwidth-\cvcolumngapwidth}
        
        % set paragraph indentation to 0, because it screws up the whole layout otherwise
        \setlength{\parindent}{0mm}
        
        
        % style definitions
        % -----------------
        % style categories explanation:
        % * \cvnameXXX is used for the name;
        % * \cvsectionXXX is used for section names (left column, accompanied by a horizontal rule);
        % * \cvtitleXXX is used for job/education titles (right column);
        % * \cvdurationXXX is used for job/education durations (left column);
        % * \cvheadingXXX is used for headings (left column);
        % * \cvmainXXX (and \setmainfont) is used for main text;
        % * \cvmarginXXX is used for text in margins;
        % * \cvruleXXX is used for the horizontal rules denoting sections.
        
        % font families
        \defaultfontfeatures{Ligatures=TeX} % reportedly a good idea, see https://tex.stackexchange.com/a/37251
        
        \newfontfamily{\cvnamefont}[Path=resources/TeX-Gyre-Heros/]{texgyreheros-regular.otf}
        \newfontfamily{\cvsectionfont}[Path=resources/TeX-Gyre-Heros/]{texgyreheros-regular.otf}
        \newfontfamily{\cvtitlefont}[Path=resources/TeX-Gyre-Heros/]{texgyreheros-regular.otf}
        \newfontfamily{\cvdurationfont}[Path=resources/TeX-Gyre-Heros/]{texgyreheros-regular.otf}
        \newfontfamily{\cvheadingfont}[Path=resources/TeX-Gyre-Heros/]{texgyreheros-regular.otf}
        \newfontfamily{\cvheadingfont}[Path=resources/TeX-Gyre-Heros/]{texgyreheros-regular.otf}
        \setmainfont[Path=resources/TeX-Gyre-Heros/]{texgyreheros-regular.otf}
        
        % colours
        \definecolor{cvnamecolor}{HTML}{3e3a38}
        \definecolor{cvsectioncolor}{HTML}{0e4194}
        \definecolor{cvtitlecolor}{HTML}{0e4194}
        \definecolor{cvdurationcolor}{HTML}{0e4194}
        \definecolor{cvheadingcolor}{HTML}{0e4194}
        \definecolor{cvmaincolor}{HTML}{3e3a38}
        \definecolor{cvmargincolor}{HTML}{1593cb}
        \definecolor{cvrulecolor}{HTML}{398dc9}
        
        \color{cvmaincolor}
        
        % styles
        \newcommand{\cvnamestyle}[1]{{\Large\cvnamefont\textcolor{cvnamecolor}{#1}}}
        \newcommand{\cvsectionstyle}[1]{{\normalsize\cvsectionfont\textcolor{cvsectioncolor}{#1}}}
        \newcommand{\cvtitlestyle}[1]{{\large\cvtitlefont\textcolor{cvtitlecolor}{#1}}}
        \newcommand{\cvdurationstyle}[1]{{\small\cvdurationfont\textcolor{cvdurationcolor}{#1}}}
        \newcommand{\cvheadingstyle}[1]{{\normalsize\cvheadingfont\textcolor{cvheadingcolor}{#1}}}
        
        
        % inter-item spacing
        % ------------------
        
        % vertical space after personal info and standard CV items
        \newlength{\cvafteritemskipamount}
        \setlength{\cvafteritemskipamount}{5mm plus 1.25mm minus 1.25mm}
        
        % vertical space after sections
        \newlength{\cvaftersectionskipamount}
        \setlength{\cvaftersectionskipamount}{2mm plus 0.5mm minus 0.5mm}
        
        % extra vertical space to be used when a section starts with an item with a heading (e.g. in the skills section),
        % so that the heading does not follow the section name too closely
        \newlength{\cvbetweensectionandheadingextraskipamount}
        \setlength{\cvbetweensectionandheadingextraskipamount}{1mm plus 0.25mm minus 0.25mm}
        
        
        % intra-item spacing
        % ------------------
        
        % vertical space after name
        \newlength{\cvafternameskipamount}
        \setlength{\cvafternameskipamount}{3mm plus 0.75mm minus 0.75mm}
        
        % vertical space after personal info lines
        \newlength{\cvafterpersonalinfolineskipamount}
        \setlength{\cvafterpersonalinfolineskipamount}{2mm plus 0.5mm minus 0.5mm}
        
        % vertical space after titles
        \newlength{\cvaftertitleskipamount}
        \setlength{\cvaftertitleskipamount}{1mm plus 0.25mm minus 0.25mm}
        
        % value to be used as parskip in right column of CV items and itemsep in lists (same for both, for consistency)
        \newlength{\cvparskip}
        \setlength{\cvparskip}{0.5mm plus 0.125mm minus 0.125mm}
        
        % set global list configuration (use parskip as itemsep, and no separation otherwise)
        \setlist{parsep=0mm,topsep=0mm,partopsep=0mm,itemsep=\cvparskip}
        
        
        % CV commands
        % -----------
        
        % creates a "personal info" CV item with the given left and right column contents, with appropriate vertical space after
        % @param #1 left column content (should be the CV photo)
        % @param #2 right column content (should be the name and personal info)
        \newcommand{\cvpersonalinfo}[2]{
            % left and right column
            \begin{minipage}[t]{\cvleftcolumnwidth}
                \vspace{0mm} % XXX hack to align to top, see https://tex.stackexchange.com/a/11632
                \raggedleft #1
            \end{minipage}% XXX necessary comment to avoid unwanted space
            \hspace{\cvcolumngapwidth}% XXX necessary comment to avoid unwanted space
            \begin{minipage}[t]{\cvrightcolumnwidth}
                \vspace{0mm} % XXX hack to align to top, see https://tex.stackexchange.com/a/11632
                #2
            \end{minipage}
        
            % space after
            \vspace{\cvafteritemskipamount}
        }
        
        % typesets a name, with appropriate vertical space after
        % @param #1 name text
        \newcommand{\cvname}[1]{
            % name
            \cvnamestyle{#1}
        
            % space after
            \vspace{\cvafternameskipamount}
        }
        
        % typesets a line of personal info beginning with an icon, with appropriate vertical space after
        % @param #1 parameters for the \includegraphics command used to include the icon
        % @param #2 icon filename
        % @param #3 line text
        \newcommand{\cvpersonalinfolinewithicon}[3]{
            % icon, vertically aligned with text (see https://tex.stackexchange.com/a/129463)
            \raisebox{.5\fontcharht\font`E-.5\height}{\includegraphics[#1]{#2}}
            % text
            #3
        
            % space after
            \vspace{\cvafterpersonalinfolineskipamount}
        }
        
        % creates a "section" CV item with the given left column content, a horizontal rule in the right column, and with
        % appropriate vertical space after
        % @param #1 left column content (should be the section name)
        \newcommand{\cvsection}[1]{
            % left and right column
            \begin{minipage}[t]{\cvleftcolumnwidth}
                \raggedleft\cvsectionstyle{#1}
            \end{minipage}% XXX necessary comment to avoid unwanted space
            \hspace{\cvcolumngapwidth}% XXX necessary comment to avoid unwanted space
            \begin{minipage}[t]{\cvrightcolumnwidth}
                \textcolor{cvrulecolor}{\rule{\cvrightcolumnwidth}{0.3mm}}
            \end{minipage}
        
            % space after
            \vspace{\cvaftersectionskipamount}
        }
        
        % creates a standard, multi-purpose CV item with the given left and right column contents, parskip set to cvparskip
        % in the right column, and with appropriate vertical space after
        % @param #1 left column content
        % @param #2 right column content
        \newcommand{\cvitem}[2]{
            % left and right column
            \begin{minipage}[t]{\cvleftcolumnwidth}
                \raggedleft #1
            \end{minipage}% XXX necessary comment to avoid unwanted space
            \hspace{\cvcolumngapwidth}% XXX necessary comment to avoid unwanted space
            \begin{minipage}[t]{\cvrightcolumnwidth}
                \setlength{\parskip}{\cvparskip} #2
            \end{minipage}
        
            % space after
            \vspace{\cvafteritemskipamount}
        }
        
        % typesets a title, with appropriate vertical space after
        % @param #1 title text
        \newcommand{\cvtitle}[1]{
            % title
            \cvtitlestyle{#1}
        
            % space after
            \vspace{\cvaftertitleskipamount}
            % XXX need to subtract cvparskip here, because it is automatically inserted after the title "paragraph"
            \vspace{-\cvparskip}
        }
        
        
        % header and footer
        % -----------------
        
        % header
        \renewcommand{\headrulewidth}{0mm} % needed to remove line under header that is there by default
        \fancypagestyle{firstpage}{
            \fancyhead[R]{} % no name in header on the first page
            \fancyhead[L]{\textcolor{cvmargincolor}{Curriculum vitae}}
        }
        \fancypagestyle{otherpages}{
            \fancyhead[R]{\textcolor{cvmargincolor}{Jeremy D Harris}}
            \fancyhead[L]{\textcolor{cvmargincolor}{Curriculum vitae}}
        }
        \thispagestyle{firstpage}
        \pagestyle{otherpages}
        
        % footer
        \pagestyle{fancy}
        \fancyfoot{} % needed to remove page number that is in centre of footer by default
        \fancyfoot[L]{\footnotesize\textcolor{cvmargincolor}{\today}}
        \fancyfoot[R]{\footnotesize\textcolor{cvmargincolor}{\thepage~/~\pageref{LastPage}}}
        
        
        
        % preamble end/document start
        % ===========================
        
        \begin{document}
        
        
        % personal info
        % -------------
        
        \cvpersonalinfo{
            % text to fill space instead of a photo
            \cvsectionstyle{PERSONAL INFORMATION}
        }{
            % name
            \cvname{Jeremy D Harris}
        
            % address
            \cvpersonalinfolinewithicon{height=4mm}{resources/europasscv-icons/address_europass_icon.pdf}{
        1325 McLendon Ave NE APT 2, Atlanta, GA 30307
            }
        
            % phone number
            \cvpersonalinfolinewithicon{height=4mm}{resources/europasscv-icons/mobile_europass_icon.pdf}{
                (717)\,329\,-\,7892
            }
        
            % email address
            \cvpersonalinfolinewithicon{height=4mm}{resources/europasscv-icons/mail_europass_icon.pdf}{
                jeremy.harris@gatech.edu
            }
        
        %% personal website
        %\cvpersonalinfolinewithicon{height=4mm}{resources/europasscv-icons/website_europass_icon.pdf}{
        %www.fake-name-homepage.com
        %}
        
        %    % date of birth
        %    Born 14 March 1985
        }
        
        
        % work experience
        % ---------------
        
        \cvsection{WORK EXPERIENCE}
        
        % Current position at GA Tech - Weitz lab
        \cvitem{
            \cvdurationstyle{June 2020 -- present}
        }{
            \cvtitle{Postdoctoral Fellow Researcher in \href{https://weitzgroup.biosci.gatech.edu/}{Weitz Group}}
        
        	Department of Biological Sciences, Georgia Tech%201 Dowman Drive Atlanta, Ga. 30322
        
            \begin{itemize}[leftmargin=*]
            	\item Advisor: Joshua Weitz, PhD, Associate Professor
                \item Aim: We aim to develop and analyze mathematical models of virus-microbe dynamics. We aim to better understand the long-term interactions of viruses and their hosts, particularly in the context of of non-lytic modes of infection such as lysogeny.
        
            \end{itemize}
        }
        
        % Current position at Emory - Koelle lab
        \cvitem{
            \cvdurationstyle{August 2017 -- May 2020}
        }{
            \cvtitle{Postdoctoral Fellow Researcher in \href{https://scholarblogs.emory.edu/koellelab/}{Koelle Research Group}}
        
        	Department of Biology, Emory University %201 Dowman Drive Atlanta, Ga. 30322
        
            \begin{itemize}[leftmargin=*]
            	\item Advisor: Katia Koelle, PhD, Associate Professor
                \item Developed mathematical models of influenza virus. In collaboration with experimental virologists, we studied the effects of multiplicity of infection (MOI) on influenza viral infection outcomes in cell culture. We also modeled a serial passage study in which varying MOIs of wild-type and defective interfering particles gave rise to cycling dynamics. Lastly, we developed a within-host stochastic exponential growth with mutations model to estimate between-host bottleneck sizes based on the number of de novo variants in recipient hosts.
        
            \end{itemize}
        }
        
        %% Fake Company 1
        %\cvitem{
        %    \cvdurationstyle{July 2010 -- September 2013}
        %}{
        %    \cvtitle{Fake junior position}
        %
        %    Fake Company 1, Fake City
        %
        %    \begin{itemize}[leftmargin=*]
        %        \item fake job description (duis aute irure dolor in reprehenderit in voluptate velit esse cillum dolore eu
        %              fugiat nulla pariatur)
        %        \item fake job details (excepteur sint occaecat cupidatat non proident)
        %    \end{itemize}
        %}
        
        
        % education
        % ---------
        
        \cvsection{EDUCATION}
        \vspace{\cvbetweensectionandheadingextraskipamount}
        
        % PhD
        \cvitem{
            \cvdurationstyle{2017}
        }{
            \cvtitle{Ph.D. in Mathematics}
        	
        	University of Pittsburgh, Pittsburgh, PA \\
            	Advisor: Bard Ermentrout, PhD, Distinguished University Professor
        
            \begin{itemize}[leftmargin=*]
                \item Thesis: ``Analysis of a spatially-distributed Wilson-Cowan model of cortex"
            \end{itemize}
        }
        
        % bachelor's
        \cvitem{
            \cvdurationstyle{2011}
        }{
            \cvtitle{B.S. in Mathematics, minor in Bioengineering}
        	University of Pittsburgh, Pittsburgh, PA
            \begin{itemize}[leftmargin=*]
                \item  Honors College, Graduated Summa Cum Laude
            \end{itemize}
        }
        
        % publications
        \cvsection{PUBLICATIONS}
        %\vspace{\cvbetweensectionandheadingextraskipamount}
        %\vspace{-1cm}
        
        \cvitem{
            \cvdurationstyle{In preparation.}
        }{
        %\vspace{-.25cm}
        %    \cvtitle{}
            \begin{itemize}[leftmargin=*]
            	\item Harris, J.D., Park, S.W., Dushoff, J., Weitz J.S. (2020). ``Modeling asymptomatic transmission in COVID-19." (in preparation, anticipated date of submission: July 2021)
        %	\item Harris, J.D., Johnson, K.E., Koelle, K.V. (2020). ``Estimation of transmission bottleneck sizes from \emph{de novo} genetic variation in recipient hosts." (in preparation, anticipated date of submission: Sept. 2020) \\
        	\item Harris, J.D.*, Martin, B.E*., Koelle, K.V., and Brooke, C.B. (2020). ``Influenza virus population cycles emerge from collections of variably responding cells." *authors contributed equally. (in preparation, anticipated date of submission: Sept. 2021)
            \end{itemize}
        }
        
        
        \cvitem{
            \cvdurationstyle{Published.}
        }{
        %\vspace{-.25cm}
        %    \cvtitle{}
            \begin{itemize}[leftmargin=*]
                	\item Martin, B.E.*, Harris, J.D.* et al. (2020). "Cellular co-infection can modulate the efficiency of influenza A virus production and shape the interferon response." PLoS pathogens 16.10: e1008974. *authors contributed equally \href{https://doi.org/10.1371/journal.ppat.1008974}{\underline{https://doi.org/10.1371/journal.ppat.1008974}}
        	        \item  Harris, J.D. and Ermentrout, G.B. (2018). ``Traveling waves in a spatially-distributed Wilson-Cowan model of cortex: From fronts to pulses." Physica D: Nonlinear Phenomena, 369, 30-46. \href{https://doi.org/10.1016/j.physd.2017.12.011}{\underline{https://doi.org/10.1016/j.physd.2017.12.011}}
            \end{itemize}
        }    	
        \cvitem{
            \cvdurationstyle{}
        }{	
        	\newpage
        \begin{itemize}[leftmargin=*]	
                \item Ali, R., Harris, J.D.*, and Ermentrout, G.B. (2016). ``Pattern formation in oscillatory media without lateral inhibition." Physical Review E, 94(1), 012412. *corresponding author \\ \href{https://doi.org/10.1103/PhysRevE.94.012412}{\underline{https://doi.org/10.1103/PhysRevE.94.012412}}
                \item Harris, J.D. and Ermentrout, G.B. (2015). ``Bifurcations in the Wilson--Cowan equations with nonsmooth firing rate." SIAM Journal on Applied Dynamical Systems, 14(1), 43-72. \href{https://doi.org/10.1137/140977953}{\underline{https://doi.org/10.1137/140977953}}
            \end{itemize}
        
        }
        
        
        % conference talks and invited presentations ------
        \cvsection{PRESENTATIONS}
        \vspace{\cvbetweensectionandheadingextraskipamount}
        
        \cvitem{
            \cvheadingstyle{2014 -- present}
        }{
        
        % during Emory/GA tech postdoc
            \cvtitle{Talks \& Posters (external)}
		While in Weitz lab:
            \begin{itemize}[leftmargin=*]
		\item Invited speaker -- ``Individual-level differences in symptomatic and asymptomatic transmission shape population-level dynamics of SARS-CoV-2 outbreaks." Virtual talk for University of Pennsylvania Math Bio seminar (Feb. 1, 2022)
            	\item 2-minute Rapid Talk (\href{https://github.com/Jeremy-D-Harris/poster_MIDAS2021.git}{poster}) -- ``Modeling asymptomatic transmission in COVID-19."  \href{https://midasnetwork.us/midas-network-annual-meeting-midas-2021/}{\underline{MIDAS 2021}}  (May 10-13) 
                	\item Invited speaker -- ``Modeling asymptomatic transmission in COVID-19." Virtual talk for University of Florida Math Bio seminar (Feb. 4, 2021)
        	\item Poster presentation -- ``Modeling asymptomatic transmission in COVID-19."  \href{https://sites.google.com/view/nsfstudentconference2021/home}{\underline{Student Conference on COVID-19 modeling}}  (May 28-29, 2021)
            	\item Invited speaker -- ``Modeling asymptomatic transmission in COVID-19." Virtual talk for University of Pittsburgh Math Bio seminar (Nov. 19, 2020)
        	    \end{itemize}
        	    	While in Koelle lab:
        	        \begin{itemize}[leftmargin=*] 
        
            	\item Conference talk -- ``Estimating transmission bottleneck sizes from viral variants unique to recipient hosts." Epidemics Conference 2019 (Dec. 3-6, 2019)
                \item Invited speaker -- ``Cellular co-infection increases viral production but the constituents of the output depend on frequencies of the input." Kennesaw State University, Applied 	Math seminar (Nov. 15, 2019)
               	\item Invited speaker -- ``How do defective interfering particles impact influenza virus dynamics?" University of Pittsburgh, Center for Vaccine Research (April 16, 2019)
                \item Discussion moderator -- summarized conference talks/posters and facilitated ``Big picture" discussion on the future of quantitative biology. A TMLS-sponsored conference at Emory (Jan. 16-18, 2019) 
                \item Poster presentation -- ``Complex viral dynamics emerge in vitro from collections of heterogeneously-responding infected cells." Evolution of Complex Life at Georgia Tech (May 15-17, 2019)
            \end{itemize}
            While in graduate school:
            % this is where grad school ends
            \begin{itemize}[leftmargin=*] 
            	\item Conference talk -- ``Traveling waves in a (nonsmooth) neural firing rate model." SIAM 2017 Annual Meeting (Pittsburgh, PA, July 10-14, 2017)
        	\item Conference talk -- ``Patterns and waves in a spatially-extended neural field model." SIAM 2017 Conference on Applications of Dynamical Systems \\(Snowbird, Utah, May 21-25, 2017)
            	\item Conference talk -- ``Travelling fronts and pulses in a nonsmooth neural mass model." SIAM 2015 Conference on Applications of Dynamical Systems \\(Snowbird, Utah, May 17-21, 2015)
        	\item Conference talk -- ``The Wilson-Cowan equations with nonsmooth firing rate." (George Mason University, March 20-21, 2015)
        	\item Conference talk --  ``Bifurcation analysis of the Wilson-Cowan equations with nonsmooth firing rate function." IEEE International Meeting on Analysis and Applications of Nonsmooth Systems (Como, Italy, August 10-12, 2014)
            \end{itemize}
        }
        
        % seminar presentations (local to Emory)
        \cvitem{
            \cvdurationstyle{2011 -- present}
        }{
            \cvtitle{Seminar Presentations (internal)}
            \begin{itemize}[leftmargin=*]
        	\item Presented, "Modeling asymptomatic transmission in COVID-19." Complex microbial dynamics and infections seminar (Oct. 9, 2020)
            	\item Presented, ``Data literacy in the sciences."  Academic Learning Community on data literacy,\, Emory (March 31, 2020)
            	\item Presented Rubin et al., "Revealing neural correlates of behavior without behavioral measurements." BioRxiv (2019). Theoretical Biophysics seminar, Emory (Oct. 21, 2019)
                \item Presented Shoval et al., ``Evolutionary Trade-Offs, Pareto Optimality, and the Geometry of Phenotype Space." Science (2012). Theoretical Biophysics seminar, Emory (Feb. 18, 2019)
                \item Presented on current methods of estimating transmission bottlenecks. Koelle lab meeting, Emory (Feb. 11, 2019)
                \item Presented Allesina and Levine, ``A competitive network theory of species diversity." PNAS (2011). Theoretical Biophysics seminar, Emory (Oct. 2017)
                \item Presented  ``The community ecology of influenza A defective interfering particles." Ecology and Evolution seminar, Emory (Oct. 2017)
            \end{itemize}
        % seminar presentations (at Pitt)
            \begin{itemize}[leftmargin=*]
                \item Presented ``Pattern formation in the Wilson-Cowan equations." Applied Math seminar, Pitt (September 25, 2015)
                \item Presented on ``Bifurcations of piecewise smooth flows." Colombo et al., Physica D: Nonlinear Phenomena (2012). Math Bio seminar, Pitt (Sept. 12 \& 17, 2015)
                \item Presented an introduction to rigidity in the tensegrity model -- In preparation for the guest speaker, R. Connelly, as part of a themed semester on networks. Math Bio seminar, Pitt (Sept. 17, 2014)
                \item Presented on ``Iterated Prisoner's Dilemma contains strategies that dominate any evolutionary opponent." Press and Dyson, PNAS (2012). Math Bio seminar, Pitt (March 6, 2014)
                \item Presented ``Processes taking place on networks." Math Bio seminar, Pitt (Feb. 5, 2013)
                \item Presented an existence proof of travelling waves in a shape-space model of antigenic variation in Trypanosomes. Applied Math seminar, Pitt (April 5, 2013)
                \item Presented on modelling antigenic variation in Trypanosome infections. Math Bio seminar, Pitt (Dec. 9, 2011 \& Oct. 29, 2012.) 
            \end{itemize}
        
        }
        
         
        \newpage
        %  teaching experience ------
        \cvsection{TEACHING EXPERIENCE}
        \vspace{\cvbetweensectionandheadingextraskipamount}
	
	\cvitem{
            \cvheadingstyle{2021 -- Present}
        }{
            \cvtitle{Teaching \& Mentoring -- GA Tech}
            \begin{itemize}[leftmargin=*]
                \item \bf{Foundations in Quantitative Biology} -- This course is for first-year QBioS program PhD students; the small class sizes (11 students) allowed for close interactions that grew over the semester. The class is intense for both students and instructors with two lectures, a computational lab, and a homework assignment each week, culminating in a final project. I had the opportunity to lecture for four of the weeks on organismal behavior, excitability, and movement. I also helped several students who did final projects on topics related to these.
                \item Undergraduate Mentoring -- During the fall semester 2021, I mentored a senior undergraduate student on a research project modeling variation in susceptibility and transmissibility in epidemic models. We found a few interesting results when considering potential correlations between susceptibility and transmissibility. For the spring semester 2022, we plan to write up these results along with further analysis (e.g., sensitivity). Contributed to a letter of support for graduate school applications.
                \item Graduate Mentoring -- Giving support for applications (fellowship award, workshops), graduate work (proposal presentation during fall 2021), research project (developing model framework for latency of viral infections). 
                \item Entering Mentoring training -- Workshop offered by Offices of Undergraduate Education and Graduate Education \& Faculty Development
                   \item \href{https://www.kitp.ucsb.edu/qbio}{KITP Quantitative biology summer research course} -- Held hands-on labs (using Matlab, R, and Python) to go though the exercises on eco-evolutionary models of viral dynamics  (August 9-13, 2021)
    	\item \href{https://workshop2021.qbios.gatech.edu/schedule/}{Quantitative Biosciences Workshop 2021: Epidemics} -- Ran a hands-on breakout session using Matlab to go though the exercises; \href{https://github.com/Jeremy-D-Harris/qBiosWorkshop2021_TeamOutbreak}{see material} (May 17-18, 2021)
    	\item Undergraduate Research Symposium -- volunteered to serve as a judge of 5-minute talks (April 22, 2021)

            \end{itemize}
        }
                
        \cvitem{
            \cvheadingstyle{2015 -- 2017}
        }{
            \cvtitle{Teaching -- Pitt}
            \begin{itemize}[leftmargin=*]
                \item Graduate Linear Algebra -- Teaching Assistant
                	\item Intro to Finite and Discrete Mathematics -- Instructor
        	\item Intro to Real Analysis -- Teaching Assistant
            	\item Intro to College Algebra -- Instructor
            	\item Calculus I (2 sections) -- Instructor
        	\item Intro to College Algebra (2 sections) -- Instructor
        	\item Intro to College Algebra -- Instructor
        	\item Calculus III -- Teaching Assistant
            \end{itemize}
        
        }
        
        \cvitem{
            \cvheadingstyle{2015 -- 2016}
        }{
            \cvtitle{Teaching Assistant Workshops -- Pitt}
            \begin{itemize}[leftmargin=*]
                \item Topics include: developing a teaching philosophy, syllabus construction, encouraging participation, teaching with Powerpoint, navigating difficult situations
            \end{itemize}
            
        }
        
        
        %\newpage
        %  service ------
        \cvsection{SERVICE}
        \vspace{\cvbetweensectionandheadingextraskipamount}
        
        \cvitem{
            \cvheadingstyle{2015 -- present}
        }{
            \cvtitle{Journal manuscript reviews}
            \begin{itemize}[leftmargin=*]
                	\item K{\"a}hne, M., R{\"u}diger, S., Kihara, A. H., and Lindner, B. (2019). Gap junctions set the speed and nucleation rate of stage I retinal waves. PLoS computational biology, 15(4), e1006355.
        	\item Nielsen, B. F. (2017). ``Regularization of Ill-posed point neuron models." The Journal of Mathematical Neuroscience, 7(1), 6.
        	\item Ji, Y., Zhang, X., Liang, M., Hua, T., and Wang, Y. (2015). ``Dynamical analysis of periodic bursting in piece-wise linear planar neuron model." Cognitive Neurodynamics, 9(6), 573-579.
        %	\item current projects include -- mixed-use paths; retro-fitting ultraviolet mesh on windows to decrease bird window strikes; 
            \end{itemize}
        
        }
    
%\cvitem{
%    \cvheadingstyle{Spring 2021 - present}
%}{
%    \cvtitle{GA Tech }
%            \begin{itemize}[leftmargin=*]
%    \item \href{https://www.kitp.ucsb.edu/qbio}{KITP Quantitative biology summer research course} -- Held hands-on labs (using Matlab, R, and Python) to go though the exercises on eco-evolutionary models of viral dynamics  (August 9-13, 2021)
%	\item \href{https://workshop2021.qbios.gatech.edu/schedule/}{Quantitative Biosciences Workshop 2021: Epidemics} -- Ran a hands-on breakout session using Matlab to go though the exercises; \href{https://github.com/Jeremy-D-Harris/qBiosWorkshop2021_TeamOutbreak}{see material} (May 17-18, 2021)
%	\item Undergraduate Research Symposium -- volunteered to serve as a judge of 5-minute talks (April 22, 2021)
%
%	        \end{itemize}
    	
%	}
            
                
        \cvitem{
            \cvheadingstyle{2018 -- Spring 2020}
        }{
            \cvtitle{Emory }
        
            \begin{itemize}[leftmargin=*]
            	\item \href{http://cfde.emory.edu/news-events/news/2019/november/data-literacy-alc.html}{Data Literacy Academic Learning Community} -- six 1.5 hour discussions on data literacy, with a focus on interdisciplinary educational approaches, skill-building in data literacy; deliverables include a package on data literacy for lessons and curriculum to support undergraduate education
        
        	\item \href{https://data-science-for-scientists-atl.github.io/}{Data Science for Scientists ATL} -- monthly meetings and special sessions on all things data (e.g. Jupyter notebook demos, version control with git, visualization with R)
        	
        	\item \href{https://data-science-for-scientists-atl.github.io/2019-11-23-emory/}{Software carpentry workshop}  -- hosted by Data Science for Scientists ATL (Nov. 23-24, 2019) -- to learn and help others learn basic shell commands, version control with git, and to use jupyter notebooks and some basic python code
        	
            	\item Datafest at Emory -- undergraduates analyze a large dataset as part of the quantitative theory and methods initiative (April 2019) 
        		
            	\item \href{http://biomed.emory.edu/news-events/events/dsac-research-symposium.html}{Graduate Research Symposium} -- helped with judging of research talks/posters (2018-2020) 	
        
        	\item Volunteer for Atlanta Bike Emory: participating in Emory Cares International Service Day (Saturday Nov. 9, 2019)
    	
    	    	\item Committee on Environment -- We discuss, review, and make recommendations on campus projects and initiatives that have an environmental impact on campus (\href{https://www.senate.emory.edu/about/committees/environment.html}{committee website}) (2019-2020)
        
            \end{itemize}
        
        }
        
        
            
        \cvitem{
            \cvheadingstyle{2018 -- Spring 2020}
        }{
            \cvtitle{Pitt}
        
            \begin{itemize}[leftmargin=*]
    	\item Representative from the math department in general body meetings; planned and organized graduate student events, including socials and the new student teaching orientation (2013 -- 2017)
    	\item Organized for graduate students as an opportunity to practice presenting their work (2014 -- 2015)
    	\item Volunteer at Pitt's Integration Bees -- Helped with the undergraduate bee (2014 \& 2015); high school bee (2015 \& 2016)
        
            \end{itemize}
        
        }
        
        
          \cvitem{
            \cvheadingstyle{April 2020 -- present}
        }{
            \cvtitle{Hearts to Nourish Hope Food Bank}
        
            \begin{itemize}[leftmargin=*]
            	\item  Volunteer through Hands on Atlanta -- monthly, from April-August 2020
        
            \end{itemize}
        
        }
        
        
        \cvitem{
            \cvheadingstyle{2018 -- 2019}
        }{
            \cvtitle{Human Rights Campaign (HRC)}
        
            \begin{itemize}[leftmargin=*]
            	\item Volunteer for HRC Atlanta Pride Brunch (Oct. 13, 2019)
        
            \end{itemize}
        
        }
        
          \cvitem{
            \cvheadingstyle{2013 -- 2014}
        }{
            \cvtitle{Volunteer for Neighborhood Learning Alliance (NLA)}
            \begin{itemize}[leftmargin=*]
               	\item Helped high school students complete online coursework to obtain equivalent
        credit for a failed or incomplete course-requirement, Pittsburgh, PA (July and August of 2013 \& 2014).
            \end{itemize}  
        }
        
        
      
    
        
        %\newpage
        %  groups ------
        \cvsection{GROUPS \& ORGANIZATIONS}
        \vspace{\cvbetweensectionandheadingextraskipamount}
        
        \cvitem{
            \cvheadingstyle{2019 -- Fall 2020}
        }{
            \cvtitle{Emory, UGA, GA Tech}
            \begin{itemize}[leftmargin=*]
                	\item \href{https://docs.google.com/document/d/1jMU-Rc--9MpwKwmoHNcR_SD4T0128dSvHaiJtECwO50/edit?usp=sharing}{Emory-UGA-GATech SARS-CoV-2 journal club} (co-organizer) --  to collect, organize, and share information on high output of COVID-19 papers. We read papers anywhere from epidemiological data analysis to vaccine efficacy studies. We meet weekly, and the google doc is kept up-to-date -- first meeting April 20; updated August 14.  \\
            	\item \href{https://emorypda.wordpress.com/postdoc-newsletter/}{Postdoctoral Science Magazine} (editor) --  to highlight research being done at Emory University and other research institutions in Atlanta; develop skills in communicating science; \href{https://emorypda.wordpress.com/2020/03/25/working-remotely-here-are-some-tips-to-stay-safe-and-productive-during-a-pandemic/}{blog post (Emory PDA) on working remotely} -- March 25, 2020; last updated December 2020 \\
            \end{itemize}
        }
        
        \cvitem{
            \cvheadingstyle{2017 -- Spring 2020}
        }{
            \cvtitle{Emory University}
            \begin{itemize}[leftmargin=*]
                \item Biology Postdoctoral Cohort -- created to build social and professional connections amongst postdocs in biology and related areas \\
                \item \href{http://livingtheory.emory.edu/}{Theory and Modeling of Living Systems (TMLS) Initiative} \\
                \item \href{https://darwinanddavis.github.io/EmoRyCodingClub/index.html}{EmoRy R \& coding club} -- to learn R studio, R markdown, version control with git \\
                \item \href{https://emory.campuslabs.com/engage/organization/data-science-for-scientists-atl}{Data Science for Scientists ATL} -- to engage with the data science community at Emory, both learning and helping with events (meetings, workshops, etc.) \\
                \item \href{http://cfde.emory.edu/news-events/news/2019/november/data-literacy-alc.html}{Data Literacy Academic Learning Community} -- seeks to provide a space for discussion and exploration of data literacy, with a focus on interdisciplinary educational approaches, skill-building in data literacy (6 meetings during spring semester 2020); one of our main goals is to develop a data literacy package that includes lessons and curriculum to support undergraduate education
            \end{itemize}
        }
        
        
        %  memberships ------
        \cvsection{CURRENT PROFESSIONAL MEMBERSHIPS}
        
        \cvitem{
            \cvheadingstyle{}
        }{
            \cvtitle{National/international Organizations}
            \begin{itemize}[leftmargin=*]
                \item Society for Industrial and Applied Mathematics (\href{https://www.siam.org/}{SIAM}) 
            	\item Society for Mathematical Biology (\href{https://www.smb.org/}{SMB})
                \item American Mathematical Society (\href{https://www.ams.org/home/page}{AMS})
        %        \item National Society of Collegiate Scholars (NSCS)
        %        \item Golden Key International Honour Society
                \item Models of infectious disease agent study (\href{https://midasnetwork.us/}{MIDAS})
            \end{itemize}
            
        }
        
        
        %%  skills ------
        %\cvsection{SKILLS}
        %\vspace{\cvbetweensectionandheadingextraskipamount}
        %
        %% computer skills
        %\cvitem{
        %    \cvheadingstyle{Languages}
        %}{
        %    Fake language -- fake proficiency description
        %    \begin{itemize}
        %        \item fake certificate description
        %    \end{itemize}
        %
        %    Another fake language -- fake proficiency description
        %}
        
        
        
        % awards and funding since undergraduate
        \cvsection{AWARDS \& FELLOWSHIPS}
        \vspace{\cvbetweensectionandheadingextraskipamount}
        
        \cvitem{
            \cvheadingstyle{2011 -- 2013}
        }{
             \cvtitle{NSF-RTG, Complex Biological Systems Group}
            \begin{itemize}[leftmargin=*]
        
                \item Complex biological systems across multiple space and time scales \\ Award number 0739261
                \item Funding for the first two years of graduate school
            \end{itemize}
            
        }
        
        
        \cvitem{
            \cvheadingstyle{April 2011}
        }{
            \cvtitle{Culver Award for undergraduate research, Department of Mathematics}
            \begin{itemize}[leftmargin=*]
        	\item For work on modeling antigenic variation in Trypanosome infections
            \end{itemize}
            
        }
        
        
        \cvitem{
            \cvheadingstyle{2010 -- 2011}
        }{
             \cvtitle{NSF-RTG, Complex Biological Systems Group}
            \begin{itemize}[leftmargin=*]
        
                \item Complex biological systems across multiple space and time scales \\ Award number 0739261
                \item Undergraduate research experience: summer (2010) \& spring/summer (2011)
            \end{itemize}
            
        }
        
        
        \cvitem{
            \cvheadingstyle{2007 -- 2011}
        }{
                \cvtitle{University of Pittsburgh scholarships} 
                \begin{itemize}[leftmargin=*]
                \item University of Pittsburgh, 2007-2011
                \item Swanson School of Engineering, 2007-2009
                \end{itemize}
        }
        
       
        % funding since undergraduate
        %\newpage
        \cvsection{FUNDING ACKNOWLEDGMENTS}
        \vspace{\cvbetweensectionandheadingextraskipamount}
        
        \cvitem{
            \cvheadingstyle{August 2017 -- present}
        }{
            \cvtitle{DARPA INTERCEPT W911NF-17-2-0034} 
            Principal Investigator: Chris B. Brooke, PhD, Assistant Professor
            \begin{itemize}
        	\item Funding source of my postdoctoral fellowship
            	\item As part of the INTERCEPT program, our research team has aimed to investigate the potential for defective interfering particles to be used as a novel therapeutic against viral infections by understanding their basic evolutionary consequences within- and between-hosts.
            \end{itemize}
            
        }
        
        \cvitem{
            \cvheadingstyle{2013 -- 2014}
        }{
            \cvtitle{NSF DMS 1219753} 
            Principal Investigator: G. Bard Ermentrout, PhD, Distinguished University Professor
            \begin{itemize}
                \item Interactions between Stimuli and Spatiotemporal Activity
                \item Working with an undergraduate REU student (summer 2014), we published our results in PRE (2016). (see publications section) \\
            \end{itemize}
            
        }
        
        
        %\newpage 
        \cvsection{REFERENCES}
        
        \begin{tabbing}
        \hskip 1in \= \href{https://scholarblogs.emory.edu/koellelab/}{Katia Koelle}              \hskip 1in \= \href{https://www.pitt.edu/~phase/}{Bard Ermentrout}     \hskip 1.2 in \= \href{http://www.math.pitt.edu/~bdoiron/}{Brent Doiron} \\    
        \>  Associate Professor													\> Distinguished University Professor			\> Associate Professor\\
        \> Department of Biology           											\> Department of Mathematics			        		\> Department of Neurobiology \\
        \> Emory University	            												\> University of Pittsburgh 				        \> University of Chicago \\
        \> 1510 Clifton Road NE                											\> 301 Thackeray Hall				                 \>bdoiron@uchicago.edu     \\
        \> Atlanta, GA 30322														\> Pittsburgh, PA  15260                     			\>   \\
        \> (404) 727-6292														\> (412) 624-8324        			                 	\>  \\   
        \> katia.koelle@emory.edu												\> bard@pitt.edu                       	    				\> \\
        %\vspace{-.9in}
        \end{tabbing}
        
        \begin{tabbing}
        \hskip 1in \= \href{https://scholarblogs.emory.edu/koellelab/}{Chris B. Brooke} \\
        \>  Assistant Professor \\
        \> Department of Microbiology \\
        \> Carl R. Woese Institute for Genomic Biology \\
        \> University of Illinois at Urbana-Champaign \\	      
        \> 601 S. Goodwin Ave \\
        \> Urbana, IL 61801 \\
        \> (217) 265-0991 \\
        \> cbrooke@illinois.edu \\
        %\vspace{-.9in}
        \end{tabbing}
        
        %% fake skills
        %\cvitem{
        %    \cvheadingstyle{Fake skills}
        %}{
        %    Fake skill 1
        %    \begin{itemize}
        %        \item fake skill description (excepteur sint occaecat cupidatat non proident)
        %        \item fake skill details (sunt in culpa qui officia deserunt mollit anim id est laborum)
        %    \end{itemize}
        %
        %    Fake skill 2
        %
        %    Fake skill 3
        %}
        %
        %% completely fake skills
        %\cvitem{
        %    \cvheadingstyle{Completely fake skills}
        %}{
        %    Completely fake skill 1
        %    \begin{itemize}
        %        \item completely fake skill description
        %    \end{itemize}
        %
        %    Completely fake skill 2
        %}
        
        
        % fake section
        % ------------
        
        %\cvsection{FAKE SECTION}
        %
        %\vspace{\cvbetweensectionandheadingextraskipamount}
        %
        %% fake stuff
        %\cvitem{
        %    \cvheadingstyle{Fake stuff}
        %}{
        %    Fake stuff description (lorem ipsum dolor sit amet, consectetur adipiscing elit, sed do eiusmod tempor incididunt
        %    ut labore et dolore magna aliqua)
        %
        %    Fake stuff details (ut enim ad minim veniam, quis nostrud exercitation ullamco laboris nisi ut aliquip ex ea
        %    commodo consequat)
        %}
        %
        %% super fake stuff
        %\cvitem{
        %    \cvheadingstyle{Super fake stuff}
        %}{
        %    Super fake stuff description (duis aute irure dolor in reprehenderit in voluptate velit esse cillum dolore eu
        %    fugiat nulla pariatur)
        %}
        
        
        % additional info
        % ---------------
        
        %\cvsection{ADDITIONAL INFORMATION}
        %
        %\vspace{\cvbetweensectionandheadingextraskipamount}
        %
        %% driving licence
        %\cvitem{
        %    \cvheadingstyle{Driving licence}
        %}{
        %    Fake category
        %}
        %
        %% interests
        %\cvitem{
        %    \cvheadingstyle{Interests}
        %}{
        %    Fake interest 1, fake interest 2, fake interest 3
        %}
        
        \end{document}
